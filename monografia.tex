% Monografia-LaTeX (Unochapec�)                                           %
% Copyright (C) 2013  Daniel Girotto                                      %
%                                                                         %
% This program is free software: you can redistribute it and/or modify    %
% it under the terms of the GNU General Public License as published by    %
% the Free Software Foundation, either version 3 of the License, or       %
% (at your option) any later version.                                     %
%                                                                         %
% This program is distributed in the hope that it will be useful,         %
% but WITHOUT ANY WARRANTY; without even the implied warranty of          %
% MERCHANTABILITY or FITNESS FOR A PARTICULAR PURPOSE.  See the           %
% GNU General Public License for more details.                            %
%                                                                         %
% You should have received a copy of the GNU General Public License       %
% along with this program.  If not, see <http://www.gnu.org/licenses/>.   %

% ------------------------------------------------------------ %
% Monografia
% ------------------------------------------------------------ %
\documentclass[pnumromarab, normaltoc, a4paper, 12pt]{abnt}

\usepackage{url}
\usepackage{tocloft}
\usepackage{leading}
\usepackage{lmodern}
\usepackage{textcomp}
\usepackage{multicol}
\usepackage{graphicx}
\usepackage{abnt-alf}
\usepackage{varwidth}
\usepackage{remreset}
\usepackage[T1]{fontenc}
\usepackage[normalem]{ulem}
\usepackage[latin1]{inputenc}
\usepackage[polutonikogreek,brazil]{babel}

\usepackage[svgnames]{xcolor}

\makeatletter
\@removefromreset{footnote}{chapter}
\makeatother

% Para que ele entenda o @
\makeatletter

% Altera o tamanho das fontes dos cap�tulos e dos ap�ndices
\renewcommand{\ABNTchapterfont}{\bfseries}
\renewcommand{\ABNTchaptersize}{\Large}
\renewcommand{\ABNTanapsize}{\Large}

% Altera o espa�amento entre dots
\renewcommand\@dotsep{0.5}

% Altera forma de montagem do TOC
\renewcommand\l@chapter[2]{
  \ifnum \c@tocdepth >\m@ne
    \addpenalty{-\@highpenalty}%
    \vskip 0em \@plus\p@
    \setlength\@tempdima{.85em}%
    \begingroup
      \ifthenelse{\boolean{ABNTpagenumstyle}}
        {\renewcommand{\@pnumwidth}{1.7em}}
        {}
      \parindent \z@ \rightskip \@pnumwidth
      \parfillskip -\@pnumwidth
      \leavevmode \normalsize\ABNTtocchapterfont
      \advance\leftskip\@tempdima
      \hskip -\leftskip
      #1\nobreak\dotfill \nobreak%
      \ifthenelse{\boolean{ABNTpagenumstyle}}
         {%
         \hb@xt@\@pnumwidth{\hss
            \ifthenelse{\not\equal{#2}{}}{{\normalfont p.\thinspace#2}}{}}\par
         }
         {%
          \hb@xt@\@pnumwidth{\hss #2}\par
         }
      \penalty\@highpenalty
    \endgroup
  \fi
}

\renewcommand*\l@section{\@dottedtocline{1}{0em}{1.6em}}
\renewcommand*\l@subsection{\@dottedtocline{2}{0em}{2.4em}}

\let\stdl@section\l@section
\renewcommand*{\l@section}[2]{%
  \leading{1mm}
  \vspace{-1.48mm}
  \stdl@section{\textcolor{black}{#1}}{\textcolor{black}{#2}}}

\let\stdl@subsection\l@subsection
\renewcommand*{\l@subsection}[2]{%
  \leading{1mm}
  \vspace{-1.48mm}
  \stdl@subsection{\it \textcolor{black}{#1}}{\textcolor{black}{#2}}
}

% Cria um comando auxiliar para montagem da lista de figuras
\newcommand{\figfillnum}[1]{%
  {\hspace{1em}\normalfont\dotfill}\nobreak
  \hb@xt@\@pnumwidth{\hfil\normalfont #1}{}\par}

% Cria um comando auxiliar para montagem da lista de tabelas
\newcommand{\tabfillnum}[1]{%
    {\hspace{1em}\normalfont\dotfill}\nobreak
    \hb@xt@\@pnumwidth{\hfil\normalfont #1}{}\par}

% Altera a forma de montagem da lista de figuras
\renewcommand*{\l@figure}[2]{
    \leftskip 3.1em
    \rightskip 1.6em
    \parfillskip -\rightskip
    \parindent 0em
    \@tempdima 2.0em
    \advance\leftskip \@tempdima \null\nobreak\hskip -\leftskip
    {FIGURA \normalfont #1}\nobreak \figfillnum{#2}}

% Altera a forma de montagem de lista de tabelas
\renewcommand*{\l@table}[2]{
    \leftskip 3.4em
    \rightskip 1.6em
    \parfillskip -\rightskip
    \parindent 0em
    \@tempdima 2.0em
    \advance\leftskip \@tempdima \null\nobreak\hskip -\leftskip
    {Tabela \normalfont #1}\nobreak \tabfillnum{#2}}

% Define o comando que monta a lista de siglas
\newcommand{\listadesiglas}{\pretextualchapter{LISTA DE SIGLAS}\@starttoc{lsg}}
\newcommand{\sigla}[2]{{\addcontentsline{lsg}{sigla}{\numberline{#1}{#2}}}#1}
\newcommand{\l@sigla}[2]{
    \vspace{-0.75cm}
    \leftskip 0em
    \parindent 0em
    \@tempdima 5em
    \advance\leftskip \@tempdima \null\nobreak\hskip -\leftskip
    {\normalfont #1}\hfil\nobreak\par}

% Define o tipo de numera��o das p�ginas
\renewcommand{\chaptertitlepagestyle}{plain}

% Altera a posi��o da numera��o de p�ginas dos elementos textuais
\renewcommand{\ABNTchaptermark}[1]{
    \ifthenelse{\boolean{ABNTNextOutOfTOC}}
        {\markboth{\ABNTnextmark}{\ABNTnextmark}}
        {\chaptermark{#1}
        \pagestyle{\chaptertitlepagestyle}}}

% Redefine o tipo de numera��o das p�ginas
\renewcommand{\ABNTBeginOfTextualPart}{
    \renewcommand{\chaptertitlepagestyle}{plainheader}
    \renewcommand{\thepage}{\arabic{page}}
    \setcounter{page}{1}}

\makeatother

% Altera o tamanho do par�grafo
\setlength{\parindent}{1.5cm}

\begin{document}

% <one line to give the program's name and a brief idea of what it does.> %
% Copyright (C) 2013  Daniel Girotto                                      %
%                                                                         %
% This program is free software: you can redistribute it and/or modify    %
% it under the terms of the GNU General Public License as published by    %
% the Free Software Foundation, either version 3 of the License, or       %
% (at your option) any later version.                                     %
%                                                                         %
% This program is distributed in the hope that it will be useful,         %
% but WITHOUT ANY WARRANTY; without even the implied warranty of          %
% MERCHANTABILITY or FITNESS FOR A PARTICULAR PURPOSE.  See the           %
% GNU General Public License for more details.                            %
%                                                                         %
% You should have received a copy of the GNU General Public License       %
% along with this program.  If not, see <http://www.gnu.org/licenses/>.   %

% ------------------------------------------------------------ %
% Capa
% ------------------------------------------------------------ %
\begin{titlepage}
  \begin{center}
    \includegraphics[scale=0.70]{images/logotipo.jpg}\\

    \textbf{UNIVERSIDADE COMUNIT�RIA DA REGI�O DE CHAPEC�}\\
    \leading{6mm}
    \textbf{�REA DE CI�NCIAS EXATAS E AMBIENTAIS}\\
    \leading{6mm}
    \textbf{CURSO DE CI�NCIA DA COMPUTA��O (OU SISTEMAS DE INFORMA��O)}\\
    \leading{6mm}
    \textbf{(BACHARELADO)}\\

    \vspace{4cm}
    \begin{center}
      \textbf{T�TULO DO TRABALHO DE CONCLUS�O DO CURSO}
    \end{center}

    \vspace{5cm}
    \begin{center}
      \textbf{NOME COMPLETO DO ALUNO}
    \end{center}

    \vspace{5cm}
    \normalfont{\textbf{CHAPEC�,}}
    \normalfont{\textbf{M�S (EXTENSO) DE ANO (4 D�GITOS)}}\\
  \end{center}
\end{titlepage}

% ------------------------------------------------------------ %
% Folha de rosto
% ------------------------------------------------------------ %
\thispagestyle{empty}
\begin{titlepage}
  \begin{center}
    \textbf{UNIVERSIDADE COMUNIT�RIA DA REGI�O DE CHAPEC�}\\
    \leading{5mm}
    \textbf{�REA DE CI�NCIAS EXATAS E AMBIENTAIS}\\
    \leading{5mm}
    \textbf{CURSO DE CI�NCIA DA COMPUTA��O (OU SISTEMAS DE INFORMA��O)}\\
    \leading{5mm}
    \textbf{(BACHARELADO)}\\
  \end{center}

  \vspace{3cm}
  \begin{center}
    \textbf{T�TULO DO TRABALHO DE CONCLUS�O DO CURSO}
  \end{center}

  \vspace{3cm}
  \hspace{0.4\textwidth}
  \begin{varwidth}{0.5\textwidth}
  \leading{5mm}
    \textbf{Relat�rio do Trabalho de Conclus�o de Curso submetido � Universidade
    \mbox{Comunit�ria} da Regi�o de Chapec� para obten��o do t�tulo de
     bacharelado no curso de Ci�ncia da Computa��o (Sistemas de Informa��o).
    }\par
  \end{varwidth}

  \vspace{0.5cm}
  \begin{center}
    \textbf{NOME COMPLETO DO ALUNO}
  \end{center}

  \hspace{.4\textwidth}
  \begin{minipage}{.5\textwidth}
    \leading{5mm}
    Orientador(a): Prof(a). Nome do(a) orientador(a) de conte�do.\par
  \end{minipage}

  \begin{center}
    \vspace{5cm}
    \normalfont{\textbf{CHAPEC�,}}
    \normalfont{\textbf{M�S (EXTENSO) DE ANO (4 D�GITOS)}}\\
  \end{center}
\end{titlepage}
% <one line to give the program's name and a brief idea of what it does.> %
% Copyright (C) 2013  Daniel Girotto                                      %
%                                                                         %
% This program is free software: you can redistribute it and/or modify    %
% it under the terms of the GNU General Public License as published by    %
% the Free Software Foundation, either version 3 of the License, or       %
% (at your option) any later version.                                     %
%                                                                         %
% This program is distributed in the hope that it will be useful,         %
% but WITHOUT ANY WARRANTY; without even the implied warranty of          %
% MERCHANTABILITY or FITNESS FOR A PARTICULAR PURPOSE.  See the           %
% GNU General Public License for more details.                            %
%                                                                         %
% You should have received a copy of the GNU General Public License       %
% along with this program.  If not, see <http://www.gnu.org/licenses/>.   %

% ------------------------------------------------------------ %
% Folha de aprova��o                                           %
% ------------------------------------------------------------ %
\begin{folhadeaprovacao}

  \begin{center}
    \textbf{T�TULO DO TRABALHO DE CONCLUS�O DO CURSO}
  \end{center}

  \vspace{0.1cm}
  \begin{center}
    \textbf{NOME COMPLETO DO ALUNO}
  \end{center}

  \begin{center}
    \leading{6mm}
    \textbf{ESTE RELAT�RIO, DO TRABALHO DE CONCLUS�O DE CURSO, FOI JULGADO
    ADEQUADO PARA OBTEN��O DO T�TULO DE:}
  \end{center}

  \begin{center}
    \leading{6mm}
    \textbf{BACHAREL EM CI�NCIA DA COMPUTA��O (OU SISTEMAS DE INFORMA��O)}
  \end{center}

  \vspace{-1.5cm}
  \hspace*{-1.5cm}
  \assinatura*{Prof. Fulano de Tal, Esp.,MSc., Dr.\\
    \textbf{Orientador}}
  \assinatura*{Prof. Outro Fulano de Tal, Esp.,MSc., Dr.\\
    \textbf{Co-Orientador}}

  \vspace{1.1cm}
  \begin{flushleft}
    \textbf{BANCA EXAMINADORA:}
  \end{flushleft}

  \vspace{-1.5cm}
  \hspace*{-1.5cm}
  \assinatura*{Prof. Fulano da Silva, Esp.,MSc., Dr.\\
    \textbf{Institui��o}}
  \assinatura*{Prof. Outro Fulano, Esp.,MSc., Dr.\\
    \textbf{Institui��o}}

  \vspace{-0.8cm}
  \hspace*{-1.5cm}
  \assinatura*{Prof. Fulano da Silva, Esp.,MSc., Dr.\\
    \textbf{ Supervisor de TCC}}
  \assinatura*{Prof. Outro Fulano, Esp.,MSc., Dr.\\
    \textbf{Coordenador de Curso}}

  \begin{center}
    \vspace{4cm}
    \normalfont{\textbf{Chapec�, }}
    \normalfont{Dia de M�s de Ano (data da defesa)}\\
  \end{center}
\end{folhadeaprovacao}
% <one line to give the program's name and a brief idea of what it does.> %
% Copyright (C) 2013  Daniel Girotto                                      %
%                                                                         %
% This program is free software: you can redistribute it and/or modify    %
% it under the terms of the GNU General Public License as published by    %
% the Free Software Foundation, either version 3 of the License, or       %
% (at your option) any later version.                                     %
%                                                                         %
% This program is distributed in the hope that it will be useful,         %
% but WITHOUT ANY WARRANTY; without even the implied warranty of          %
% MERCHANTABILITY or FITNESS FOR A PARTICULAR PURPOSE.  See the           %
% GNU General Public License for more details.                            %
%                                                                         %
% You should have received a copy of the GNU General Public License       %
% along with this program.  If not, see <http://www.gnu.org/licenses/>.   %

% ------------------------------------------------------------ %
% Dedicat�ria
% ------------------------------------------------------------ %
\vspace*{19.8cm} 
\hspace{0.4\textwidth}
\begin{varwidth}{0.48\textwidth}
  \leading{5mm}
  Dedico.....\\
  Item OPCIONAL, deve ficar posicionado ao final da folha.\\
  � uma men��o onde o autor presta \mbox{homenagem} ou dedica o trabalho a
  algu�m
\end{varwidth}
% Monografia-LaTeX (Unochapec�)                                           %
% Copyright (C) 2013  Daniel Girotto                                      %
%                                                                         %
% This program is free software: you can redistribute it and/or modify    %
% it under the terms of the GNU General Public License as published by    %
% the Free Software Foundation, either version 3 of the License, or       %
% (at your option) any later version.                                     %
%                                                                         %
% This program is distributed in the hope that it will be useful,         %
% but WITHOUT ANY WARRANTY; without even the implied warranty of          %
% MERCHANTABILITY or FITNESS FOR A PARTICULAR PURPOSE.  See the           %
% GNU General Public License for more details.                            %
%                                                                         %
% You should have received a copy of the GNU General Public License       %
% along with this program.  If not, see <http://www.gnu.org/licenses/>.   %

% ------------------------------------------------------------ %
% Agradecimento
% ------------------------------------------------------------ %
\vspace*{18.5cm}
\hspace{0.4\textwidth}
\begin{varwidth}{0.48\textwidth}
  \leading{5mm}
  Agrade�o....\\
  Item OPCIONAL, deve ficar posicionado ao final da folha.\\
  S�o men��es a pessoas e/ou institui��es das quais eventualmente recebeu apoio
  e que concorreram de maneira relevante para o desenvolvimento do trabalho.
\end{varwidth}
% Monografia-LaTeX (Unochapec�)                                           %
% Copyright (C) 2013  Daniel Girotto                                      %
%                                                                         %
% This program is free software: you can redistribute it and/or modify    %
% it under the terms of the GNU General Public License as published by    %
% the Free Software Foundation, either version 3 of the License, or       %
% (at your option) any later version.                                     %
%                                                                         %
% This program is distributed in the hope that it will be useful,         %
% but WITHOUT ANY WARRANTY; without even the implied warranty of          %
% MERCHANTABILITY or FITNESS FOR A PARTICULAR PURPOSE.  See the           %
% GNU General Public License for more details.                            %
%                                                                         %
% You should have received a copy of the GNU General Public License       %
% along with this program.  If not, see <http://www.gnu.org/licenses/>.   %

% ------------------------------------------------------------ %
% Ep�grafe
% ------------------------------------------------------------ %
\vspace*{18.5cm}
\hspace{0.4\textwidth}
\begin{varwidth}{0.5\textwidth}
  \leading{5mm}
  \raggedleft
  Ep�grafe\\
  Item OPCIONAL, deve ficar posicionado ao final da folha.\\
  � a inscri��o de um trecho em prosa ou composi��o po�tica que de certa forma
  embasou a constru��o do trabalho, seguida da indica��o de autoria.\\
  Fulano de tal
\end{varwidth}
% Monografia-LaTeX (Unochapec�)                                           %
% Copyright (C) 2013  Daniel Girotto                                      %
%                                                                         %
% This program is free software: you can redistribute it and/or modify    %
% it under the terms of the GNU General Public License as published by    %
% the Free Software Foundation, either version 3 of the License, or       %
% (at your option) any later version.                                     %
%                                                                         %
% This program is distributed in the hope that it will be useful,         %
% but WITHOUT ANY WARRANTY; without even the implied warranty of          %
% MERCHANTABILITY or FITNESS FOR A PARTICULAR PURPOSE.  See the           %
% GNU General Public License for more details.                            %
%                                                                         %
% You should have received a copy of the GNU General Public License       %
% along with this program.  If not, see <http://www.gnu.org/licenses/>.   %

% ------------------------------------------------------------ %
% Resumo
% ------------------------------------------------------------ %
\thispagestyle{empty}
\begin{center}
  \vspace*{.85cm}
  \textbf{RESUMO}
\end{center}

\vspace{2.5cm}
\noindent
\leading{5.5mm}
� a apresenta��o concisa do texto, destacando seus aspectos de maior relev�ncia.
� redigido em um �nico par�grafo, contendo entre 250 e 500 palavras. Usar
terceira pessoa do singular. N�o usar cita��es bibliogr�ficas. Ressaltar
objetivos, m�todos, resultados e conclus�es do trabalho.

\vspace{1.3cm}

\noindent
\textbf{Palavras-chave:} entre tr�s e cinco palavras-chave, separadas por ponto
e v�rgula.
% Monografia-LaTeX (Unochapec�)                                           %
% Copyright (C) 2013  Daniel Girotto                                      %
%                                                                         %
% This program is free software: you can redistribute it and/or modify    %
% it under the terms of the GNU General Public License as published by    %
% the Free Software Foundation, either version 3 of the License, or       %
% (at your option) any later version.                                     %
%                                                                         %
% This program is distributed in the hope that it will be useful,         %
% but WITHOUT ANY WARRANTY; without even the implied warranty of          %
% MERCHANTABILITY or FITNESS FOR A PARTICULAR PURPOSE.  See the           %
% GNU General Public License for more details.                            %
%                                                                         %
% You should have received a copy of the GNU General Public License       %
% along with this program.  If not, see <http://www.gnu.org/licenses/>.   %

% ------------------------------------------------------------ %
% Abstract
% ------------------------------------------------------------ %
\thispagestyle{empty}
\begin{center}
  \vspace*{.95cm}
  \textbf{ABSTRACT}
\end{center}

\vspace{2.5cm}
\noindent
\leading{5.5mm}
It is a brief presentation of the text in English, where the most relevant
aspects are underlined. It should be written in the third person singular, in an
only paragraph containing from 250 to 500 words. You should not mention
references. It is important to write the objectives, method, results and the
final remarks of the study.

\vspace{1.3cm}

\noindent
\textbf{Keywords:} from 3 to 5 keywords, separated by semicolon.

\addcontentsline{toc}{chapter}{LISTA DE ILUSTRA��ES}

% Altera o t�tulo do TOC
\renewcommand{\contentsname}{\vspace*{.7cm}
  \normalsize{\textbf{SUM�RIO}}\vspace{1.5cm}}

\sumario
\listoffigures
\listadesiglas
\addcontentsline{toc}{chapter}{RESUMO}
\addcontentsline{toc}{chapter}{ABSTRACT}

\chapter{INTRODU��O}
\section{Contextualiza��o}
\section{Delimita��o do problema}
\section{Justificativa}
\section{Objetivos}
\subsection{Objetivo geral}
\subsection{Objetivos espec�ficos}
\section{Procedimentos Metodol�gicos}
\section{Estrutura do Trabalho}

\chapter{NORMAS PARA DIGITA��O DO TRABALHO}

\begin{figure}[hbtp]
\begin{center}
\includegraphics[width=100mm]{images/logotipo.jpg}
\caption{http://publicdomainreview.org/2012/07/09/arabic-machine-manuscript}
\end{center}
\end{figure}

\section{Apresenta��o}
\section{Elementos Textuais}
\subsection{Introdu��o}
\subsection{Desenvolvimento}
\subsection{Conclus�es e trabalhos futuros}
\section{Elementos P�s-Textuais}
\subsection{Refer�ncias bibliogr�ficas}
\subsection{Bibliografia complementar}
\subsection{Ap�ndices}
\subsection{Anexos}
\section{Elementos de apoio ao texto}
\section{Editora��o}
\subsection{Formata��o de estilos}
\section{Conclus�o}

\begin{table}[ht]
\caption{Lemas mais usados por C�cero (cima pra baixo, esquerda pra direita)}
\begin{center}
\begin{tabular}{|c|c|c|c|c|c|}
\hline
dico & uerbum & quantus & oportet & lex & anima \\ \hline
possum & populus & suis & auctoritas & modius & liber \\ \hline
uideo & uita & ciuis & quaero & ars & exercitus \\ \hline
homo & animus & suo & facilis & arbitror & reliquus \\ \hline
facio & tuus & suus & prouincia & quin & praetor \\ \hline
causa & scribo & sententia & studium & genus & honor \\ \hline
habeo & multus & semper & loco & uirus & diligo \\ \hline
uolo & lego & nullus & intellego & plus & bene \\ \hline
iudex & consilium & ceterus & corpus & periculum & bellum \\ \hline
tantus & ius & saepe & dignitas & optimus & sentio \\ \hline
bonus & oratio & maior & pater & soleo & domus \\ \hline
ratio & inquam & uir & dolor & scriba & sapio \\ \hline
primus & dies & totus & scio & opus & rego \\ \hline
publica & gener & pars & do & dica & diu \\ \hline
maximus & uirtus & iudicium & ago & modus & imperium \\ \hline
tempus & pono & ueneo & necesse & proficiscor & plures \\ \hline
littera & itaque & pecunia & audio & mors & accipio \\ \hline
nunc & deus & minor & umquam & spes & praesertim \\ \hline
senatus & locus & debeo & fortuna & urbs & uno \\ \hline
uis & ciuitas & numquam & ullus & pario & malis \\ \hline
multa & consul & fors & omnino & duo & uoluntas \\ \hline
natura & satis & summo & orator & gratia & amicus \\ \hline
puto & nomen & animo & salus & memoria & moueo \\ \hline
magnus & publico & potis & utor & credo & mens \\ \hline
alter & solus & fero & uoluptas & populor & postea \\ \hline
\end{tabular}
\end{center}
\end{table}

\end{document}